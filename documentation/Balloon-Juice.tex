\documentclass{article}

\usepackage{mathrsfs,amsmath}
\usepackage{xcolor}
\usepackage{titlesec}
\usepackage{mwe}
\usepackage{graphicx}

\usepackage[margin=1.4in]{geometry}

\title{Project 3 | Balloon Juice}
\author{Jared Dyreson \\
        jareddyreson@csu.fullerton.edu \\
        Mason Godfrey \\
        mgodfrey@csu.fullerton.edu \\
        California State University, Fullerton}
\date

\usepackage [english]{babel}
\usepackage [autostyle, english = american]{csquotes}
\MakeOuterQuote{"}

\titlespacing*{\section}
{0pt}{5.5ex plus 1ex minus .2ex}{4.3ex plus .2ex}
\titlespacing*{\subsection}
{0pt}{5.5ex plus 1ex minus .2ex}{4.3ex plus .2ex}

\usepackage{hyperref}
\hypersetup{
    colorlinks,
    citecolor=black,
    filecolor=black,
    linkcolor=black,
    urlcolor=blue
}

\graphicspath{ {./assets/} }

\begin{document}

\maketitle
\tableofcontents

\newpage

\section{Precursor Information}

\begin{enumerate}
\item Members: Mason Godfrey, Jared Dyreson
\item Class: CS-335, Section 02
\item Team Name: Jurrito
\end{enumerate}

\section{Introduction}

This project focuses on a mission reconnaissance visualizer, showing a bot retrieve a balloon in a deeply nested balloon field.
The mission is to traverse from balloon to balloon, leaving markers of where it has been in the process.
Our bot has a very simple set of instructions and rules it must adhere to.
Those are as follows:

\begin{enumerate}
\item The bot must move forward once and cannot repeat such action after the fact. That path is considered dead.
\item The bot can go backwards and follows the same principle as the rule above.
\item Teleportation is not permitted
\item The bot will finally show it's path taken in reverse order, making it's way back to the initial balloon.
\end{enumerate}

\begin{flushleft}
In this project, we have the balloon field initialized to an $V \times V$ adjacency matrix, where $V$ is defined as the number of balloons occupying the field (here we have 40).
\end{flushleft}

\subsection{Floyd-Warshall Algorithm}

The algorithm will find the shortest path given a matrix with weighted edges, where the connection spanning from $u \rightarrow v$ signifies an edge or connection.
These edge values are then used to find the lengths of said shortest path.
By design, the algorithm does not attempt to reconstruct the path but in our rendition it does.
The average running time of this algorithm is $\theta(|V^3|)$.

\newpage

\section{Archive Contents}

\subsection{documentation}
\begin{itemize}
\item exported/Balloon-Juice.pdf | PDF rendition of our documentation
\item Balloon-Juice.tex | LaTeX version of our documentation
\item Balloon-Juice.txt | plain text version of our documentation (directly converted from above using pandoc). If viewing in plain text, please consider checking out the PDF for the images to be displayed in line. Otherwise, please navigate to "assets" to see them; nomenclature of the file names should  be self-explanatory.
\item assets/\* | all images pertaining to this documentation piece
\end{itemize}

\subsection{root}

\begin{itemize}
\item Display.js | All drawing functionality that is conducted by p5
\item index.html | webpage to display all the drawing
\item README.md | Small readme about the project (not related to the one found in the readme)
\item RuntimeAnalysis.txt | plain text version of our running time analysis
\end{itemize}

\subsection{runtime}

\begin{itemize}
\item assets/p5.js | this file is considered an import and has been given by Professor Siska
\item backend/Bot.js | contains logic for the bot that traverses the balloon field (remove randomGenerator)
\item backend/GenPoint.js | randomly generates balloons in the balloon field, ensuring they have proper DPV values
\item graphics/Balloon.js | a balloon and balloon field class implementation that controls the graphical movement of the balloons
\item graphics/Cell.js | a graphical representation of the balloons that get displayed onto the screen
\item graphics/DPV.js | helps control the logic behind the generation of balloons
\end{itemize}

\newpage

\section{Installation and Running}

\begin{enumerate}
\item Extract the .zip file into a folder
\item Navigate to the driver.html file within the folder. Right-click on the file and select “Open”. The p5 program should start immediately. A field of balloons should appear and the algorithm should start displaying the path taken.
\end{enumerate}

\subsection{Icon Legend}

\begin{center}
\large \textbf{Starting Node}

\fbox{\includegraphics[width=2cm]{start-node.png}}

\end{center}

\vspace*{7px}

\begin{center}
\large \textbf{Ending Node}

\fbox{\includegraphics[width=2cm]{end-node.png}}

\end{center}

\section{Known Bugs}

Currently there are no known bugs to exist nor undesired program outcomes.
However, we were not able to complete the table visualization of our Floyd-Warshal algorithm as there were several overlapping bugs and no time to address.
Therefore we have printed the entire traversal through the console and can be examined there.


\section{Testing}

The team has not experience any warnings or ill side effects from running Balloon-Juice.

\newpage

\section{Credits}

\begin{flushleft}
\textbf{Jared:}

\begin{enumerate}
\item \href{https://en.wikipedia.org/wiki/Floyd-Warshall_algorithm}{Floyd-Warshall Algorithm}
\end{enumerate}
\end{flushleft}

\end{document}
